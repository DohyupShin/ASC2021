%%%%%%%%%%%%%%%%%%%%%%%%%%%%%%%%%%%%%%%%%%%%%%%%%%%%%%%%%%%%%%%%%%%%%%%%%%%%%%%%%%%%
% Do not alter this block (unless you're familiar with LaTeX
\documentclass{article}
\usepackage[left=2cm,right=2cm,top=2.54cm,bottom=2cm,a4paper]{geometry}
\usepackage[onehalfspacing]{setspace}
\usepackage{amsmath,amsthm,amssymb,amsfonts, fancyhdr, color, comment, graphicx, environ}
\usepackage{xcolor}
\usepackage{mdframed}
\usepackage[shortlabels]{enumitem}
\usepackage{indentfirst}
\usepackage{hyperref}
\hypersetup{
    colorlinks=true,
    linkcolor=blue,
    filecolor=magenta,      
    urlcolor=blue,
}

\pagestyle{fancy}

\newtheorem{theorem}{Theorem}
\newtheorem{lemma}{Lemma}
\newtheorem{assumption}{Assumption}
\newtheorem{proposition}{Proposition}
\newtheorem{definition}{Definition}

\theoremstyle{definition}
\newtheorem{thm}{Theorem}[section]
\newtheorem{lem}[thm]{Lemma}
\newtheorem{prop}[thm]{Proposition}
\newtheorem{cor}[thm]{Corollary}

\theoremstyle{definition}
\newtheorem*{defn}{Definition}
\newtheorem*{exmp}{Example}

\theoremstyle{remark}
\newtheorem*{rem}{Remark}
\newtheorem*{rec}{Recall}
\newtheorem*{note}{Note}
\newtheorem*{exer}{Exercise}
\newtheorem*{fact}{Fact}



%overset
\newcommand{\ovs}[2]{\overset{#1}{#2}}


\newcommand{\mat}[1]{\vec{#1}}
\newcommand{\set}[1]{\mathcal{#1}}
% transpose
\newcommand{\T}{\mathsf{T}} 
% inner product 
\newcommand{\ip}[2]{\left< #1,\, #2 \right>} 
% absolute value 
\newcommand{\abs}[1]{\left\lvert #1 \right\rvert} 
% norm 
\newcommand{\norm}[1]{\left\lVert #1 \right\rVert} 
% bar over text (to replace \bar and \overline)
\newcommand{\paren}[1]{\left( #1 \right)}

\newcommand{\mybar}[1]{\mkern 1.5mu \overline{\mkern -1.5mu #1 \mkern -1.5mu} \mkern 1.5mu} 
% hat over text (to replace \hat and \widehat)
\newcommand{\myhat}[1]{\mkern 1.5mu \widehat{\mkern -1.5mu #1 \mkern -1.5mu} \mkern 1.5mu} 
% tilde over text (to replace $\tilde and \widetilde)
\newcommand{\mytilde}[1]{\mkern 1.5mu \widetilde{\mkern -1.5mu #1 \mkern -1.5mu} \mkern 1.5mu}


% the set of all natural numbers 
\newcommand{\N}{\mathbb{N}} 
% the set of all real numbers 
\newcommand{\R}{\mathbb{R}} 
% the set of all real n-dimmensional vectors
\newcommand{\Rn}{\mathbb{R}^{n}} 
% the set of all complex numbers 
\newcommand{\C}{\mathbb{C}} 
% the set of all complex n-dimmensional vectors
\newcommand{\Cn}{\mathbb{C}^{n}}



\newenvironment{problem}[2][Problem]
    { \begin{mdframed}[backgroundcolor=gray!20] \textbf{#1 #2} \\}
    {  \end{mdframed}}

% Define solution environment
\newenvironment{solution}{\textbf{Solution}}

%%%%%%%%%%%%%%%%%%%%%%%%%%%%%%%%%%%%%%%%%%%%%
%Fill in the appropriate information below
\lhead{2021-28605}
\rhead{Dohyup Shin} 
\chead{\textbf{Advanced Statistical Method HW 4}}
%%%%%%%%%%%%%%%%%%%%%%%%%%%%%%%%%%%%%%%%%%%%%


\begin{document}

    \begin{problem}{5.6}
	If $x \sim Mult_L (n, \pi)$, use the Poisson trick (5.44) to appropriate the mean and variance of $x_1/x_2$. (Here we are assumming that $n\pi_2$ is large enough to ignore the possibility $x_2 = 0$.)
    Hint: In notation (5.41),
    $$ \frac{S1}{S2} \doteq \frac{\mu_1}{\mu_2}(1 + \frac{S_1 - \mu_1}{\mu_1} - \frac{S_2 - \mu_2}{\mu_2}).$$
    \begin{align*}
      S_l \ovs{ind}{\sim} Poi(\mu_l), \quad l = 1,2, \dots, L \quad \quad (5.41)&\\
      Mult_L(N, \pi) \sim Poi(n\pi)  \quad \quad (5.44)&  
    \end{align*}



    \end{problem}
    
    \noindent
    \begin{solution}
    Let $X = (x_1, x_2, \dots, x_L)\sim Mult_L(n, \mathbf{\pi})$ where $\mathbf{\pi} = (\pi_1, \pi_2, \dots, \pi_L)$ and $N \sim Poi(n)$. Then, by using Poisson trick, we can approximate $X = (x_1, x_2, \dots, x_L) \sim Poi(n\mathbf{\pi})$. In other words, $x_i \ovs{indep}{\sim} Poi(n\pi_i) \; \forall i = 1, 2, \dots, L$. Define $\mu_i = n\pi_i \; \forall i$. We know that $E_{\pi}(x_i) = n\pi_i = \mu_i \text{ and } Var_{\pi} = n\pi_i = \mu_i$. Next,  we use the hint given to the problem. Then, we can calculate the mean and variance of $x_1/x_2$.

    First, the mean of $x_1/x_2$ is
    \begin{align*}
            E_{\mathbf{\pi}}(\frac{x_1}{x_2}) & = E_{\mathbf{\pi}} \paren{\frac{\mu_1}{\mu_2}(1 + \frac{x_1 - \mu_1}{\mu_1} - \frac{x_2 - \mu_2}{\mu_2}} \\[6pt]
            & = \frac{\mu_1}{\mu_2}(1 + E_{\mathbf{\pi}}\paren{\frac{x_1 - \mu_1}{\mu_1}} - E_{\mathbf{\pi}} \paren{\frac{x_2 - \mu_2}{\mu_2}})\\[6pt]
            & = \frac{\mu_1}{\mu_2}
    \end{align*}

    Second, the variance of $x_1/x_2$ is
    \begin{align*}
        Var_{\mathbf{\pi}}(\frac{x_1}{x_2}) & = Var_{\mathbf{\pi}} \paren{\frac{\mu_1}{\mu_2}(1 + \frac{x_1 - \mu_1}{\mu_1} - \frac{x_2 - \mu_2}{\mu_2}} \\[6pt]
        & = \frac{\mu_1^2}{\mu_2^2}(\frac{1}{\mu_1^2} Var_{\mathbf{\pi}}(x_1) + \frac{1}{\mu_2^2} Var_{\mathbf{\pi}}(x_2))\\[6pt]
        & = \frac{\mu_1^2}{\mu_2^2}(\frac{1}{\mu_1} + \frac{1}{\mu_2})\\[6pt]
        & = \frac{\mu_1(\mu_1 + \mu_2)}{\mu_2^3}
\end{align*}

    
    \end{solution}
    
  
    \begin{problem}{5.7}
    Show explicitly how the binomial density bi(12, 0.3) is an exponential tilt of bi(12, 0.6).
    \end{problem}
    
    \noindent
    \begin{solution}

    
    \end{solution}

  

\end{document}